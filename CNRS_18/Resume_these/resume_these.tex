\documentclass[paper=a4, fontsize=12pt,twoside]{scrartcl}
\usepackage{enumerate}
\usepackage{url,graphicx,tabularx,array,geometry}
\usepackage[explicit]{titlesec}
\usepackage{caption}
\usepackage{sectsty} % Allows customizing section commands
\usepackage[normalem]{ulem}
\usepackage{wrapfig}
%\sectionfont{\color{red}{}\fontfamily{phv}\selectfont\underline}
\newcommand{\colorulem}[1][black]{\bgroup
\ifdim\ULdepth=\maxdimen\settodepth\ULdepth{(j}\advance\ULdepth.4pt\fi
\markoverwith{\kern0em\vtop{\kern\ULdepth {\color{#1}\hrule width .4em}}\kern0em}\ULon}
 %  \sectionfont{\sffamily\itshape}
    %\sectionfont{\MakeUppercase\rmfamily\center\underline}
\usepackage[T1]{fontenc} % Use 8-bit encoding that has 256 glyphs
\usepackage{color} 
\definecolor{heading}{rgb}{0.5,1,0}
\usepackage{fourier} % Use the Adobe Utopia font for the document - comment this line to return to the LaTeX default
\usepackage[english]{babel} % English language/hyphenation
\usepackage{amsmath,amsfonts,amsthm} % Math packages
\usepackage{hyperref}
\hypersetup{
    colorlinks,%
    citecolor=black,%
    filecolor=black,%
    linkcolor=black,%
    urlcolor=blue     % can put red here to visualize the links
}

\setlength{\parskip}{1ex} %--skip lines between paragraphs

%\usepackage{fancyhdr} % Custom headers and footers
%%\pagestyle{fancyplain} % Makes all pages in the document conform to the custom headers and footers
%%\fancyhead{} % No page header - if you want one, create it in the same way as the footers below
%\fancyhead{}
%\fancyfoot{}
%\fancyhf{}
%\pagestyle{fancy}
%%%%%%%%%%%%%%%%%%%%%%%%%%%%% The paper headers
%\fancyhead[RO,LE]{\small\thepage}
%\fancyhead[LO]{\small \textsc{El Mellah} Ileyk}% odd page header and number to right top
%%\fancyhead[RE]{\small Degenerate stars}%Even page header and number at left top
%\fancyfoot[L,R,C]{}
%%\fancyfoot[C]{} % Empty center footer
%%\fancyfoot[R]{\thepage} % Page numbering for right footer
%
%\renewcommand{\footrulewidth}{0pt} % Remove footer underlines
%\setlength{\headheight}{13.6pt} % Customize the height of the header

\pagenumbering{gobble}

\geometry{
 a4paper,
 total={170mm,257mm},
 left=20mm,
 top=20mm,
 }

\renewcommand*\rmdefault{iwona}

\renewcommand \thesection{\colorulem[red]{\Roman{section}}}
\renewcommand \thesubsection{\Roman{section}.\colorulem[green]{\arabic{subsection}}}

% -------------------------------------------------
\newcommand{\horrule}[1]{\rule{\linewidth}{#1}} % Create horizontal rule command with 1 argument of height

\title{	
\vspace*{-1.5cm}
\normalfont \normalsize 
\textsc{El Mellah Ileyk} \\ [25pt] % Your university, school and/or department name(s)
\horrule{0.5pt} \\[0.4cm] % Thin top horizontal rule
\huge Wind accretion onto\\ compact objects \\ % The assignment title
\horrule{2pt} \\[0.5cm] % Thick bottom horizontal rule
}
%
%\author{\textsc{El Mellah} Ileyk} % Your name
%
\date{} % Today's date or a custom date
% -------------------------------------------------


\begin{document}

\maketitle
%%
\thispagestyle{empty}

%\title{Compact objects}
%\line
%\leftright{\today}{Ileyk E{\sc l mellah}} %-- left and right positions in the header
\vspace*{-3cm}
\subsubsection*{Abstract}

X-ray emission associated to accretion onto compact objects displays important levels of photometric
and spectroscopic time-variability. When the accretor orbits a Supergiant star, it captures a fraction of
the supersonic radiatively-driven wind which forms shocks in its vicinity. The amplitude and stability of
this gravitational beaming of the flow conditions the mass accretion rate responsible, in fine, for the X-ray
luminosity of those Supergiant X-ray Binaries. The capacity of this low angular momentum inflow to form
a disc-like structure susceptible to be the stage of well-known instabilities remains at stake.
Using state-of-the-art numerical setups, we characterized the structure of a Bondi-Hoyle-Lyttleton
flow onto a compact object, from the shock down to the vicinity of the accretor, typically five orders of
magnitude smaller. The evolution of the mass accretion rate and of the bow shock which forms around
the accretor (transverse structure, opening angle, stability, temperature profile...) with the Mach number
of the incoming flow is described in detail. The robustness of those simulations based on the High
Performance Computing \texttt{MPI-AMRVAC} code is supported by the topology of the inner sonic surface, in
agreement with theoretical expectations.\\

We developed a synthetic model of mass transfer in Supergiant X-ray Binaries which couples the launching
of the wind accordingly to the stellar parameters, the orbital evolution of the streamlines in a modified
Roche potential and the accretion process. We show that the shape of the permanent flow is entirely determined
by the mass ratio, the filling factor, the Eddington factor and the alpha force multiplier. Provided
scales such as the orbital period are known, we can trace back the observables to evaluate the mass accretion
rates, the accretion mechanism (stream or wind-dominated) and the shearing of the inflow, tracer of
its capacity to form a disc around the accretor.\\

%\textbf{Aims}
\subsubsection*{PhD committee}

Pr\'esident : Dr St\'ephane Corbel (AIM CEA Saclay)\\
Directeur : Dr Andrea Goldwurm (APC Paris 7)\\
Co-directeur : Dr Fabien Casse (APC Paris 7)\\
Rapporteurs : Pr Maximilian Ruffert (University of Edinburgh)\\
\phantom{Rapporteurs :} Dr Guillaume Dubus (IPAG)\\
Examinateurs : Dr Thierry Foglizzo (AIM CEA Saclay)\\
\phantom{Examinateurs :} Pr Rony Keppens (KU Leuven)\\

%\begin{itemize}
%\item[] Pr\'esident : Dr St\'ephane Corbel (AIM CEA Saclay)
%\item[] Directeur : Dr Andrea Goldwurm (APC Paris 7)
%\item[] Co-directeur : Dr Fabien Casse (APC Paris 7)
%\item[] Rapporteurs : Pr Maximilian Ruffert (University of Edinburgh)
%\item[]               Dr Guillaume Dubus (IPAG)
%\item[] Examinateurs : Dr Thierry Foglizzo (AIM CEA Saclay)
%\item[]                Pr Rony Keppens (KU Leuven)
%\end{itemize}

% --------------------------------------------------

\end{document}