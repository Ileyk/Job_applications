 \documentclass[12pt]{letter}
 \usepackage[T1]{fontenc}
\usepackage[utf8]{inputenc}
\usepackage{fancyhdr}
\usepackage{lastpage}

%\usepackage{geometry}
\usepackage[paper=a4paper,textwidth=140mm,left=2.cm,right=2.cm,top=0.8cm,bottom=0.8cm]{geometry}

%\pagestyle{fancy}
%\lhead{\footnotesize \parbox{11cm}{Draft 1} }
%\lfoot{\footnotesize \parbox{11cm}{\textit{2}}}\cfoot{}
%\rhead{\footnotesize 3}
%\rfoot{\footnotesize Page \thepage\ of \pageref{LastPage}}
%\renewcommand{\headheight}{24pt}
%\renewcommand{\footrulewidth}{0.4pt}

\renewcommand*\rmdefault{iwona}

\pagenumbering{gobble}

\address{\phantom{a}}
       

      
\begin{document}
 
%\signature{Ileyk E\sc{l mellah}} 
 
\begin{letter} {}% Madame la Proviseure \\ Lyc\'ee  C{\sc harlemagne} \\ 14 rue Charlemagne
     % \\ 75004 {\sc paris}}
      
\newgeometry{left=3cm,right=3cm,top=8.5cm,bottom=3cm}


\date{January 21\textsuperscript{st}, 2019}
%\vspace*{-2cm}
\opening{To the members of the CNAP committee,}
 
\thispagestyle{empty}
 
%\thispagestyle{headings}
%\markright{John Smith\hfill On page styles\hfill}
 
%\thispagestyle{fancy}

\hspace*{0.5cm} I am a $[$Pegasus$]^2$ Marie Sk\l{}odowska-Curie fellow in KU Leuven, at the Center for mathematical Plasma Astrophysics (CmPA), working in Computational Astrophysics with Rony Keppens. I joined the CmPA in October 2016 after defending my PhD on \textit{Wind accretion onto compact objects}, under the supervision of Andrea Goldwurm and Fabien Casse. I apply to the position of \textit{Astronome-adjoint} at the IRAP, in the GAHEC team, for I believe my profile could match the expected requirements and since it would be a valuable support to pursue and develop further my emerging academic career. I wish to be assigned to the ANO2, \textit{Instrumentation des grands observatoires au sol et spatiaux}, in the SNO \textit{Astronomie Astrophysique}. \\ \\
\hspace*{0.5cm} After my undergraduate studies at the Ecole Normale Sup\'erieure, I volunteered to join the Kepler satellite data analysis effort under Saul Rappaport's lead at MIT. There, I was introduced to stellar evolution and binary systems and took an active part in the discovery and characterization of the first disintegrating exoplanet in 2012. My involvement also contributed to the identification of 30 new triple star systems and to a detailed analysis of the shortest-period exoplanets, those right in the spotlight of their host star. This seminal long term experience in Research laid the foundations of my scientific program : a better understanding of stellar objects and remnants in interaction with their environment.\\ \\
\hspace*{0.5cm} As I started my PhD, I turned to numerical tools to complement the analytical skills I had acquired during the previous years and model the turbulent twilight of binary systems, the X-ray binaries. I got familiar with advanced techniques such as solvers for hyperbolic partial differential equations and parallel computing, in the context of the finite volume MHD code \texttt{MPI-AMRVAC}. With several successful proposals on Tier-1 clusters and the code development I carried out, I could run the widest dynamics simulations of wind accretion onto compact objects.\\

\newpage 

\newgeometry{left=3cm,right=3cm,top=4cm,bottom=4cm}

\hspace*{0.5cm} By the end of my first postdoctoral year in KU Leuven, I was granted a 3-years $[$Pegasus$]^2$ Marie Sk\l{}odowska-Curie fellowship. I also joined an ISSI sponsored collaboration led by Silvia Mart\'{i}nez-N\'{u}\~{n}ez (IFCA) and Peter Kretschmar (ESAC) to gather observers and theoreticians from the X-ray binaries and massive stars winds communities. It enabled me to design and confront simulations of the accretion process in Supergiant X-ray binaries to the most recent observations of Vela X-1. Thanks to Jon Sundqvist and collaborators' simulations of the internal shocks in the wind of isolated massive stars, we could evaluate the impact of the wind micro-structure on the time variability of the mass accretion rate onto the neutron star.\\ \\
\hspace*{0.5cm} I am now willing to extend my investigations to the accretion/ejection process in the disc formed by the merging of a neutron star with another neutron star or a black hole. The expertise already available at the IRAP in the domain of accretion at high rates on compact objects would be a decisive asset to pursue this goal. May you judge my application admissible, I remain fully available to bring further information you might need.\\ \\
 
Sincerely,
 
\closing{Ileyk El Mellah} 


  \end{letter}
  
  
 \end{document}