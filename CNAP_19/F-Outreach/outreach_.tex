\documentclass[paper=a4, fontsize=12pt,twoside]{scrartcl}
\usepackage{enumerate}
\usepackage{url,graphicx,tabularx,array,geometry}
\usepackage[explicit]{titlesec}
\usepackage{caption}
\usepackage{sectsty} % Allows customizing section commands
\usepackage[normalem]{ulem}
\usepackage{wrapfig}
%\sectionfont{\color{red}{}\fontfamily{phv}\selectfont\underline}
\newcommand{\colorulem}[1][black]{\bgroup
\ifdim\ULdepth=\maxdimen\settodepth\ULdepth{(j}\advance\ULdepth.4pt\fi
\markoverwith{\kern0em\vtop{\kern\ULdepth {\color{#1}\hrule width .4em}}\kern0em}\ULon}
 %  \sectionfont{\sffamily\itshape}
    %\sectionfont{\MakeUppercase\rmfamily\center\underline}
\usepackage[T1]{fontenc} % Use 8-bit encoding that has 256 glyphs
\usepackage{color} 
\definecolor{heading}{rgb}{0.5,1,0}
\usepackage{fourier} % Use the Adobe Utopia font for the document - comment this line to return to the LaTeX default
\usepackage[english]{babel} % English language/hyphenation
\usepackage{amsmath,amsfonts,amsthm} % Math packages
\usepackage{hyperref}
\hypersetup{
    colorlinks,%
    citecolor=black,%
    filecolor=black,%
    linkcolor=black,%
    urlcolor=blue     % can put red here to visualize the links
}

\setlength{\parskip}{1ex} %--skip lines between paragraphs

%\usepackage{fancyhdr} % Custom headers and footers
%%\pagestyle{fancyplain} % Makes all pages in the document conform to the custom headers and footers
%%\fancyhead{} % No page header - if you want one, create it in the same way as the footers below
%\fancyhead{}
%\fancyfoot{}
%\fancyhf{}
%\pagestyle{fancy}
%%%%%%%%%%%%%%%%%%%%%%%%%%%%% The paper headers
%\fancyhead[RO,LE]{\small\thepage}
%\fancyhead[LO]{\small \textsc{El Mellah} Ileyk}% odd page header and number to right top
%%\fancyhead[RE]{\small Degenerate stars}%Even page header and number at left top
%\fancyfoot[L,R,C]{}
%%\fancyfoot[C]{} % Empty center footer
%%\fancyfoot[R]{\thepage} % Page numbering for right footer
%
%\renewcommand{\footrulewidth}{0pt} % Remove footer underlines
%\setlength{\headheight}{13.6pt} % Customize the height of the header

\pagenumbering{gobble}

\geometry{
 a4paper,
 total={170mm,257mm},
 left=20mm,
 top=20mm,
 }

%\renewcommand*\rmdefault{iwona}

\renewcommand \thesection{\colorulem[red]{\Roman{section}}}
\renewcommand \thesubsection{\Roman{section}.\colorulem[green]{\arabic{subsection}}}

% -------------------------------------------------
\newcommand{\horrule}[1]{\rule{\linewidth}{#1}} % Create horizontal rule command with 1 argument of height

\title{	
\vspace*{-1cm}
\normalfont \normalsize 
El Mellah Ileyk \\ [25pt] % Your university, school and/or department name(s)
\horrule{0.5pt} \\[0.4cm] % Thin top horizontal rule
\huge Diffusion des connaissances\\ \& vulgarisation \\ % The assignment title
\horrule{2pt} \\[0.5cm] % Thick bottom horizontal rule
}
%
%\author{\textsc{El Mellah} Ileyk} % Your name
%
\date{} % Today's date or a custom date
% -------------------------------------------------


\begin{document}

\maketitle
%%
\thispagestyle{empty}

\vspace*{-2cm}

\indent En Novembre 2017, j'ai particip\'e \`a une \'emission radiophonique, \href{https://www.mixcloud.com/faconde/faconde-s2e01-vulgarisation/}{la Faconde}, sur la radio universitaire de l'Universit\'e Libre de Bruxelles. J'y ai discut\'e de la communication scientifique et promu sa composante sensible et esth\'etique, par opposition \`a la transmission m\'ecanique de r\'esultats scientifiques format\'es qui d\'esenchante et, in fine, suscite un d\'etachement et un relativisme g\'en\'eralis\'e au sein de la population.\\

\indent Dans une perspective plus p\'edagogique, j'ai anim\'e un atelier sur la notion de potentiel en m\'ecanique \`a destination d'\'el\`eves de lyc\'ee en Octobre 2016, \`a l'occasion de la F\^ete de la Science. Pour ce faire, j'ai mis \`a profit des maquettes de potentiels de Roche que j'ai pu imprim\'ees en 3D gr\^ace \`a l'assistance technique de Hubert Halloin et Marco Agnan et \`a un financement DIM ACAV\footnote{Domaine d'Int\'er\^et Majeur en Astrophysique et Conditions d'Apparition de la Vie.}. En parall\`ele, j'ai produit une \href{http://demonstrations.wolfram.com/TrajectoryOfATestMassInARochePotential/}{application int\'eractive en ligne} qui permet d'appr\'ehender empiriquement la notion de potentiel lorsqu'elle est utilis\'ee conjointement avec la maquette suscit\'ee : la maquette sert \`a visualiser le potentiel en 3D pour un param\`etre donn\'e (le rapport de masse entre les deux corps), alors que l'application en ligne permet de modifier ce param\`etre \`a souhait et de visualiser les trajectoires associ\'ees.


% --------------------------------------------------

\end{document}