%%%%%%%%%%%%%%%%%%%%%%%%%%%%%%%%%
% Commandes pour la compilation	%
% latex cv.tex                 	%
% dvips -Ppdf -t a4 cv.dvi     	%
% ps2pdf cv.ps                 	%
%%%%%%%%%%%%%%%%%%%%%%%%%%%%%%%%%

\documentclass[a4paper,oneside]{article}

\usepackage[francais]{babel} 
\usepackage[utf8x]{inputenc}  % les accents dans le fichier.tex
\usepackage{vmargin} %left, top, right, bottom
%\usepackage{shading} % Pour griser le fond du titre pour le format Postscript
\usepackage{graphicx} % Pour insérer des images
\usepackage{wrapfig} % Pour placer les images

\setmarginsrb{2cm}{1.2cm}{2cm}{1.2cm}{0cm}{0cm}{0cm}{0cm}



%%%%%%%%%%%%%%%%%%%%%%%%%%%%%%%%%%%%
% Définition de quelques macros    %
%%%%%%%%%%%%%%%%%%%%%%%%%%%%%%%%%%%%

% ligne horizontale sur toute la page. Usage : \ligne{Largeur}
\newcommand{\ligne}[1]{\rule[0.5ex]{\textwidth}{#1}\\}
\newcommand{\interRubrique}{\bigskip}
\newcommand{\styleRub}[1]{\noindent\textbf{\large #1}\par}
\newcommand{\indentStd}{\noindent\hspace{\lenA}}



%%%%%%%%%%%%%%%%%%%%%%%%%%%%%%%%%%%
% Commandes Personnalisées    	  %
%%%%%%%%%%%%%%%%%%%%%%%%%%%%%%%%%%%

% Personnalisation du titre avec un cadre grisé.
\newcommand{\mytitle}[1]{
    \begin{center}
    {\large{\textbf{#1}\ }}
    \end{center}
}



%%%%%%%%%%%%%%%%%%%%%%%%%%%%%%%%%%%%%
% L'environnement "rubrique" 
%
% Usage : \begin{rubrique}[Indentation]{Titre} [...] \end{rubrique}
% Ensuite, la première colonne contient par exemple les dates, la seconde
% le descriptif.
% Par exemple :
%
% \begin{rubrique}{3.5cm}{pipotage}
% 1999--2000 	& ligne 1\\
% 		& ligne 2\\
% 1998--1999	& ligne 1\\
% [etc...]
% \end{rubrique}
%%%%%%%%%%%%%%%%%%%%%%%%%%%%%%%%%%%%%

\newenvironment{rubrique}[2][\linewidth] {
    \styleRub{#2}
    \setlength{\lenB}{#1}
    \setlength{\lenC}{\linewidth}
    \addtolength{\lenC}{-\lenA}
    \addtolength{\lenC}{-\lenB}
    \addtolength{\lenC}{-\parindent}
    \addtolength{\lenC}{-9pt}
    \indentStd\begin{tabular}[t]{p{\lenB}p{\lenC}}
}
{\end{tabular}}

\newenvironment{header}[2][\linewidth] {
    \styleRub{#2}
    \setlength{\lenB}{#1}
    \setlength{\lenC}{\linewidth}
    \addtolength{\lenC}{-\lenA}
    \addtolength{\lenC}{-\lenB}
    \addtolength{\lenC}{-\parindent}
    \addtolength{\lenC}{-9pt}
    \indentStd\begin{tabular}[t]{p{\lenB}ll}
}
{\end{tabular}}



%%%%%%%%%%%%%%%%%%%%%%%%%%%%%%%%%%%%%%%%%%%%
% Commandes utilisables dans le descriptif %
%					   %
% Modifiables à loisir... 		   %
%%%%%%%%%%%%%%%%%%%%%%%%%%%%%%%%%%%%%%%%%%%%

\newcommand{\lieu}[1]{\small{\textsl{#1}\ }}
\newcommand{\activite}[1]{\textbf{#1}\ }
\newcommand{\comment}[1]{\textsl{#1}\ }



%%%%%%%%%%%%%%%%%%%%%%%%%%%%%%%%%%%%%%%%%%
% Début du CV proprement dit (ouf ! :) ) %
%%%%%%%%%%%%%%%%%%%%%%%%%%%%%%%%%%%%%%%%%%

\pagestyle{empty} % pour ne pas indiquer de numéro de page...
\begin{document}
\newlength{\lenA} % indentation au début d'une ligne
\setlength{\lenA}{0.cm}
\newlength{\lenB} % Taille champ dates
\newlength{\lenC} % Taille champ description



%%%%%%%%%%%%%%%%%%%%%%%%%%%%%%%%%
% en-tête 			%
%%%%%%%%%%%%%%%%%%%%%%%%%%%%%%%%%	

\begin{header}[14cm]{}		
    \textbf{PELOTON Julien} \\
    \footnotesize {Date of Birth : 3rd November 1988} \\
    \footnotesize{Country of Citizenship : France} \\
    \footnotesize{ } \\
    \footnotesize{Laboratoire AstroParticule et Cosmologie, Universit\'e Paris Diderot} \\
    \footnotesize{10, rue Alice Domon et L\'eonie Duquet} \\
    \footnotesize{Bat. Condorcet, Case 7020} \\
    \footnotesize{75205 PARIS Cedex 13} \\
    \footnotesize{FRANCE} \\
    %\footnotesize{} \\			
    %\footnotesize{Fixe : XX.XX.XX.XX.XX} \\
    \footnotesize{Tel : +33(0)157276908} \\
    \footnotesize{E-mail : \texttt{julien.peloton at apc.univ-paris7.fr}} \\	
\end{header}

\interRubrique
\interRubrique

\mytitle{Curriculum Vitae}

\interRubrique



%%%%%%%%%%%%%%%%%%
% Bloc rubriques %
%%%%%%%%%%%%%%%%%%

\interRubrique

\begin{rubrique}[3.4cm]{Education}
\ligne{0.1mm}

2012-(2015)
& \textbf{Graduate student in Cosmology with Dr. Radek Stompor} - \lieu{Laboratoire AstroParticule et Cosmologie (APC) at the University Paris 7 - Denis Diderot, Paris} \\
& Data analysis and Scientific Exploitation of data sets of the CMB B-mode experiment, POLARBEAR. \\ \\

    2011-2012
& \textbf{Master 2 Nuclear, Particle, Astroparticle and Cosmology (NPAC)} - \lieu{Universit\'e Paris-Sud XI, Orsay, France}\\
& Year of specialization : \textbf{Cosmology}. \\ 
& Obtained with distinctions. \\ \\

 2010-2011
& \textbf{Master 1} - \lieu{Imperial College University, London, UK}\\
& Physics, ERASMUS exchange. \\ \\
    
2009-2012
& \textbf{Licence, Master and Magist\`ere in Fundamental Physics} - \lieu{Universit\'e Paris-Sud XI, Orsay}\\
& \textit{Equivalent to Bsc and Msc + 1}. \\
& Obtained with distinctions. \\ \\

2006-2009
& \textbf{Preparatory classes for competitive examinations (CPGE) in Physics} - \lieu{Marcelin Berthelot, Saint-Maur-des-Foss\'es (France)}\\
& PCSI/PSI \\ \\


\end{rubrique}

\interRubrique
%\interRubrique

\begin{rubrique}[3.4cm]{Research history}
    \ligne{0.1mm}
2012-Present
& \textbf{Data analysis and Scientific Exploitation of data sets of the CMB B-mode experiment, POLARBEAR.} - \lieu{APC, Universit\'e Paris 7}\\
& Regular visits to University of California, Berkeley (Berkeley Cosmology group and LBNL). \\
& Advisor : Dr. R. Stompor \\ \\

    2010-2011
& \textbf{BPS Brane in Supergravity} - \lieu{Physics Department, The Blackett Laboratory, Imperial College, London, UK}\\
& Advisor : Prof. K. Stelle \\ \\

    2010 (2 months)
& \textbf{Non-gaussianities in the CMB} - \lieu{Laboratoire de Physique Th\'eorique d'Orsay, Universit\'e Paris XI}\\
& Advisor : Prof. Bartjan van Tent \\ \\

\end{rubrique}

\interRubrique
%\interRubrique

\begin{rubrique}[3.4cm]{Teaching}
    \ligne{0.1mm}
    
2012-2015
& Teaching assistant at Universit\'e Paris Diderot : \\
& 2012-2015 : \textbf{Physics for 1st year at Medical School}\\
& 2013 : \textbf{Differential equation for 2nd year in Biology}\\
& 2012 : \textbf{Office automation and basics in computer science for 1st year in Physics}\\ \\

\end{rubrique}

\interRubrique

\begin{rubrique}[3.4cm]{Refereed Journal Articles}
    \ligne{0.1mm}
    
2014
& Polarbear Collaboration : \textbf{A Measurement of the Cosmic Microwave Background B-Mode Polarization Power Spectrum at Sub-Degree Scales with {\sc polarbear}} - \lieu{ApJ, \textbf{794}, 171}\\ \\

2014
& Polarbear Collaboration : \textbf{Measurement of the Cosmic Microwave Background Polarization Lensing Power Spectrum with the {\sc polarbear} experiment} - \lieu{Phys. Rev. Lett. \textbf{113}, 021301}\\ \\

2014
& Polarbear Collaboration : \textbf{Evidence for Gravitational Lensing of the Cosmic Microwave Background Polarization from Cross-correlation with the Cosmic Infrared Background} - \lieu{Phys. Rev. Lett. \textbf{112}, 131302}\\

\end{rubrique}

\interRubrique
%\interRubrique

\begin{rubrique}[3.4cm]{Conference Proceedings}
    \ligne{0.1mm}

2104
& Peloton, J. et al. : \textbf{{\sc polarbear} experiment:  Data analysis of the first season} - \lieu{The proceedings of the 49$^{th}$ Rencontres de Moriond}\\ \\

2014
& Fabbian, G. et al. : \textbf{The {\sc polarbear} experiment: first season results and beyond} - \lieu{The proceedings of the 49$^{th}$ Rencontres de Moriond}\\ \\
    
2014
& Barron, D. et al. : \textbf{The {\sc polarbear} cosmic microwave background polarization experiment} - \lieu{J. Low Temperature Physics, 2014}\\ \\

2014
& Nishino, H. et al : \textbf{{\sc polarbear} CMB Polarization Experiment} - \lieu{JPS Conf. Proc. 1, 013107 (2014)}\\ \\

2014
& Arnold, K. et al : \textbf{The {\sc Simons Array}: expanding {\sc polarbear} to three multi-chroic telescopes} - \lieu{Society of Photo-Optical Instrumentation Engineers (SPIE) Conference Series, {\bf{9153}}} \\ \\

2014
& Inoue, Y. et al. : \textbf{Thermal and optical characterization for {\sc polarbear}-2 optical system} - \lieu{Society of Photo-Optical Instrumentation Engineers (SPIE) Conference Series, {\bf{9153}}} \\ \\

2014
& Barron, D. et al. : \textbf{Development and characterization of the readout system for {\sc polarbear}-2} - \lieu{Society of Photo-Optical Instrumentation Engineers (SPIE) Conference Series, {\bf 9153}}  \\ 

\end{rubrique}

\interRubrique
%\interRubrique

\begin{rubrique}[3.4cm]{In preparation}
    \ligne{0.1mm}

2014
& A. Ferte, J. Peloton, J. Grain and R. Stompor : \textbf{Detecting the tensor-to-scalar ratio with the pure pseudospectrum reconstruction of B-mode} - \lieu{in prep.} \\

\end{rubrique}

\interRubrique

\begin{rubrique}[3.4cm]{Talks}
    \ligne{0.1mm}

August 2014
& \textbf{Collaboration meeting} - \lieu{San Diego, CA, USA}\\
& Map-making and power spectrum estimation \\ \\

May 2014
& \textbf{Cosmology seminar, IAS} - \lieu{Orsay, France}\\
& POLARBEAR experiment: Results from the first observational campaign and characterization of systematic instrumental effect. \\ \\

May 2014
& \textbf{Astronomical Polarimetry (ASTROPOL) 2014} - \lieu{Grenoble, France}\\
& Contributed talk: POLARBEAR experiment: data analysis and results of the first season \\ \\

March 2014
& \textbf{Rencontres de Moriond, Cosmology 2014} - \lieu{La Thuile, Italy}\\
& Poster: POLARBEAR experiment:  Data analysis of the first season \\ \\
    
July 2013
& \textbf{ICTP: New light in cosmology from the CMB} - \lieu{Trieste, Italy}\\
& Contributed talk: Exploring CMB polarization with POLARBEAR \\ \\

June 2013
& \textbf{LPTHE} - \lieu{Universit\'e Paris 6 - Jussieu, France}\\
& Exploring CMB polarization with POLARBEAR \\ \\

\end{rubrique}

\interRubrique
%\interRubrique

\begin{rubrique}[3.4cm]{Awards and Honors}
    \ligne{0.1mm}
2012
& PhD Fellowship from the French Ministry of Education and Research and awarded by the Graduate school Particles, Nuclei and Cosmology (PNC) and by the University Paris 7 - Denis Diderot. \\ \\

\end{rubrique}

\interRubrique
%\interRubrique

\begin{rubrique}[3.4cm]{Administrative responsibility}
    \ligne{0.1mm}
2013-2015
& Elected as the representative of the graduate students to the laboratory council. \\ \\

\end{rubrique}

\interRubrique
%\interRubrique

%\begin{rubrique}[3.4cm]{Hard and Soft skills}
%    \ligne{0.1mm}
%    French
%    & maternal \\
%    English
%    & fluent \\
%    German
%    & basic \\
%    Computer language
%    &  C, Python, IDL \\
%    Others
%    & MPI standard, HEALPix library, S2HAT library, \LaTeX,...
%\end{rubrique}

%\interRubrique
%\interRubrique

\end{document}


