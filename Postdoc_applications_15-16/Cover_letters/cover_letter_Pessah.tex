 \documentclass[12pt]{letter}
 
\usepackage{fancyhdr}
\usepackage{lastpage}

%\usepackage{geometry}
\usepackage[paper=a4paper,textwidth=140mm,left=2.5cm,right=2.5cm,top=1.4cm,bottom=0.8cm]{geometry}

%\pagestyle{fancy}
%\lhead{\footnotesize \parbox{11cm}{Draft 1} }
%\lfoot{\footnotesize \parbox{11cm}{\textit{2}}}\cfoot{}
%\rhead{\footnotesize 3}
%\rfoot{\footnotesize Page \thepage\ of \pageref{LastPage}}
%\renewcommand{\headheight}{24pt}
%\renewcommand{\footrulewidth}{0.4pt}


\address{\phantom{a}}
       

      
\begin{document}
 
%\signature{Ileyk E\sc{l mellah}} 
 
\begin{letter} {}% Madame la Proviseure \\ Lyc\'ee  C{\sc harlemagne} \\ 14 rue Charlemagne
     % \\ 75004 {\sc paris}}
      
%\newgeometry{left=2cm,right=2cm,top=2.5cm,bottom=2.5cm}


\date{28\textsuperscript{th} November, 2015}

\opening{Dear Professors,}
 
\thispagestyle{empty}
 
%\thispagestyle{headings}
%\markright{John Smith\hfill On page styles\hfill}
 
%\thispagestyle{fancy}

\hspace*{0.5cm} I am applying to the postdoctoral position in Astrophysics \& Planetary Sciences in the Theoretical Astrophysics Group of the Niels Bohr International Academy. I am currently a 3\textsuperscript{rd} year graduate student at the AstroParticule \& Cosmology laboratory in the University of Paris 7 Diderot. Since fall 2013, I have been conducting my PhD research in Computational Astrophysics under the supervision of Fabien Casse and Andrea Goldwurm and I am expected to defend in September 2016.\\
\hspace*{0.5cm} After my studies at the \textsc{ens}, I volunteered to join Saul Rappaport at \textsc{mit} in 2011-12. There, I contributed to his efforts to make the most of the Kepler satellite data for exoplanets and stellar binaries investigations. This inspiring insight into binary systems drove me into the study of one of their turbulent twilight, the X-ray binaries. Fabien Casse then convinced me of the relevance of the numerical tool to complement the analytical skills I had acquired during the previous years. Indeed, the diversity of behaviours of those systems suggests an unavoidable need to pay attention to non-linear evolutions whose full analytical derivation remains beyond our current abilities. Existing semi-analytical scenarios have remarkably succeeded in accounting for specific observational features in X-ray binaries. Yet, our difficulties to devise a unique or even unified frame of thought demonstrate the price to pay for reducing a complex system to a small enough number of parameters to handle it. This is where high performance \textsc{mhd} simulations can be game changers. The new hardware technologies (e.g. \textsc{gpu} and InfiniBand) and optimized algorithmic schemes (e.g. \textsc{amr} and flux-limited diffusion) both provided us with an incredible computing power. It is a determining moment to seize this quantitative opportunity to qualitatively supersede the previous semi-analytical models of accretion in X-ray binaries with more holistic simulations then ever before.\\
\hspace*{0.5cm} I am applying to this postdoctoral position in Copenhagen for I believe my experience and commitment to computational and theoretical Astrophysics make me well qualified to meet the needs of the Niels Bohr Institute fellowships. I want to capitalize on the numerical expertise I have acquired and develop new numerical setups able to tackle multi-scale and multi-Physics problems. I think I can fit in an enthusiastic and stimulating environment such as the Theoretical Astrophysics Group whose ; I already had the pleasure to meet two members of the group, Colin McNally and Thomas Berlok, at the Ecole des Houches.\\
\hspace*{0.5cm} I have passed the French Agr\'egation in Physics where I ranked second and was granted teaching responsibilities at the Paris 7 Diderot University for the last three years. I also actively took part in the organization of the \textit{Rencontre des Jeunes Physiciens} (Meeting of the Young Physicists) and in the promotion of Physics in festivals. I do intend to pursue my outreach and organizing activities and would gladly teach and monitor junior fellows.\\
\hspace*{0.5cm} I look forward to hearing from you.\\
 
Sincerely,
 
\closing{Ileyk E{\sc l Mellah}} 


  \end{letter}
  
  
 \end{document}