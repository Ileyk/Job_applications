\documentclass[11pt]{letter}
\usepackage[english]{babel}
\usepackage{multicol}

\setlength{\textheight}{8in}


\begin{document}
%\thispagestyle{empty}
\begin{letter}{Ariane \textsc{Trescases}\\
CMLA, ENS Cachan \& CNRS\\
61 Av. du Pdt. Wilson, 94235 Cachan , France\\
+33 6 87 69 65 87\\
{\ttfamily trescase@cmla.ens-cachan.fr}}
\date{}
\opening{}

\begin{flushright}
Paris, July 16th, 2015.
\end{flushright}

\vspace{0.8cm}
\emph{Subject: application to the position of Research Associate -- reference LF04750}

\vspace{0.8cm}
 
Dear member of the Committee,
 
\vspace{0.8cm}

I, Ms. Ariane \textsc{Trescases}, currently a PhD student at CMLA, ENS Cachan under the supervision of Prof. Laurent \textsc{Desvillettes}, am pleased to send you my application for the position of Research Associate at the Department of Pure Mathematics and Mathematical Statistics, University of Cambridge.

\vspace{0.4cm}

\quad My research deals with the analysis and modeling of partial differential equations arising in Physics, Biology and other Sciences. More specifically, my thesis focuses on the qualitative analysis of two particular cases of such models, namely the Boltzmann equation for rarefied gases confined in a box (with applications in Aerospace engineering) and Cross diffusion-reaction systems in Population dynamics, which model competitive behaviours between two species (with applications in Biology and Ecology). During my PhD, I published five papers on these two models, and I was invited to present my work to a public of international specialists in more than a dozen universities and organizations, including five international conferences and an international school.
 
\vspace{0.4cm}

\quad As part of my PhD, I taught a load of 128 hours in total in the class of Preparation to the French \emph{Agr\' egation} (provided to Master students).

\vspace{0.4cm}

\quad My research projects now include notably a systematic study of the singularities arising in a variety of kinetic equations (coming from modeling) and an extension of the methods I developed for the cross-diffusion systems in Population dynamics to more generic systems arising in Biology. The mathematical analysis will furthermore be completed by developing and performing adequate numerical simulations.
 
\vspace{0.4cm}

\quad I believe that my research absolutely fits into the topics tackled by the ERC project MATKIT; and I know I would find at the Department of Pure Mathematics and Mathematical Statistics the opportunity to regularly discuss with top researchers whose works are directly related to mine. Thanks to Dr. Amit \textsc{Einav}'s invitation to give a talk at the Geometric Analysis and PDE seminar and to visit the Department of Pure Mathematics and Mathematical Statistics for a few days in last June, I met part of the group of researchers involved in kinetic theory and I remember with enthousiasm a vivid and pleasant environment. During these days I also had the opportunity to work with Prof. Cl\'ement \textsc{Mouhot}, with whom discussions have always been inspiring. I think becoming a Research Associate at the Department of Pure Mathematics and Mathematical Statistics would definitely help me to enlarge the scope of my studies by crossing the talents of the Department's people and the skills I have developed for the last four years. In addition, I am willing to tackle administrative and logistical duties to support the Department. For example, I am willing to participate to the organisation of scientific gatherings with the help of my own network.

\vspace{0.4cm}

\quad In conclusion, I am ready to actively take part in enhancing the team work, and I have my heart set on improving the Mathematical research in such a unique working environment. As a young researcher, I would definitely get the most of the position of Research Associate at the Department of Pure Mathematics and Mathematical Statistics, which would be a decisive turn for pursuing my carreer in academic research.


\vspace{0.8cm}

I look forward to hearing from you.\\
Sincerey yours,\\
Ariane \textsc{Trescases}

\end{letter}

\end{document}

