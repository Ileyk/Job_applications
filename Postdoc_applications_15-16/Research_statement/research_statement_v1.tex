%%%%%%%%%%%%%%%%%  Debut du fichier Latex  %%%%%%%%%%%%%%%%%%%%%%%%%%%%%%
\documentclass[a4paper,12pt,onecolumn]{article}

%%% Pour un texte en francais

%%\usepackage[applemac]{inputenc}
%\usepackage[francais]{babel}
	         % encodage des lettres accentuees
\usepackage[T1]{fontenc}
\usepackage[utf8]{inputenc}          % encodage des lettres accentuees
%\usepackage{graphicx}
%%\usepackage{graphicx} \def\BIB{}
\usepackage[paper=a4paper,textwidth=140mm,left=3cm,right=3cm,top=3.1cm,bottom=3.1cm]{geometry}
\usepackage{multicol}
\usepackage{graphicx,wrapfig,lipsum} \def\BIB{}
\usepackage[pdftex]{hyperref}
\usepackage[round]{natbib}
\usepackage{perpage} %the perpage package
\MakePerPage{footnote} %the perpage package command
\hypersetup{
    colorlinks,%
    citecolor=black,%
    filecolor=black,%
    linkcolor=black,%
    urlcolor=blue     % can put red here to visualize the links
}

\DeclareUnicodeCharacter{00A0}{ }

%%% Quelques raccourcis pour la mise en page
\newcommand{\remarque}[1]{{\small \it #1}}
\newcommand{\rubrique}{\bigskip \noindent $\bullet$ }

\newcommand{\ignore}[1]{}

\pagenumbering{gobble}

%\bibliographystyle{abbrvnat}
%\setcitestyle{authoryear,open={((},close={))}}

%\renewcommand{\thefootnote}{\roman{footnote}}

% -------------------------------------------------
\newcommand{\horrule}[1]{\rule{\linewidth}{#1}} % Create horizontal rule command with 1 argument of height

\title{	
\vspace*{-2cm}
\normalfont \normalsize 
%\textsc{Paris Diderot} \\ [25pt] % Your university, school and/or department name(s)
\horrule{0.5pt} \\[0.4cm] % Thin top horizontal rule
\huge Research statement \\ % The assignment title
\horrule{2pt} \\[0.5cm] % Thick bottom horizontal rule
}

\author{E{\sc l mellah} Ileyk} % Your name

\date{\small 18 Novembre 2015}%\normalsize\today} % Today's date or a custom date
% -------------------------------------------------

%\makeatletter
%\def\@xfootnote[#1]{%
%  \protected@xdef\@thefnmark{#1}%
%  \@footnotemark\@footnotetext}
%\makeatother

\begin{document}

\bibpunct{[}{]}{;}{n}{,}{,}

%%%%%%%%%%%%%%%%%%%%%%%%%  PREMIERE PAGE %%%%%%%%%%%%%%%%%%%%%%%%%%%%%%
%%% DANS CETTE PAGE, ON REMPLACE LES INDICATIONS ENTRE CROCHETS [...]
%%% PAR LES INFORMATIONS DEMANDEES
%%%%%%%%%%%%%%%%%%%%%%%%%%%%%%%%%%%%%%%%%%%%%%%%%%%%%%%%%%%%%%%%%%%%%%%

\maketitle
\thispagestyle{empty}

\indent \indent During my PhD, I have focused my research interest on the numerical simulations of gas being accreted on to compact objects. This ubiquitous phenomenon has played a major role in the development of high energy Astrophysics for the last decades. Still today, new models of accretion are proposed to explain specific behaviours which depart from the big pictures drawn in the 70's. The wide spatial and temporal scales covered by those systems provide a unique opportunity to emphasize the diversity of landscapes that similar Physical principles can deliver.\\
\indent Historically the first confirmed extrasolar X-ray source, X-ray binaries are believed to host a stellar companion and a compact object. As the former transfers matter to the latter, gas heats up to temperatures much higher than stellar photosphere ones. Among those X-ray binaries, many different photometric and spectroscopic characteristic behaviours have been observed, defining different families, among which the Supergiant X-ray Binaries (Sg\textsc{xb}) where matter is believed to be transferred preferentially through the stellar winds of the massive stellar companion. The fast increase since the late 2000's in the number of Sg\textsc{xb}s discovered \citep{Walter15} led me to investigate the case of wind accretion, the low angular momentum counterpart of the much more comprehensively understood Roche lobe overflow (\textsc{rlof}) accretion. \ignore{Their high X-ray variability suggests more powerful instabilities, possibly at the scale of the whole system. }Since the first sketch of a compact object accreting the wind from its stellar companion \citep{Illarionov1975}, the scientific literature had troubles to agree on the specific processes responsible for the observed signatures. The discovery of Supergiant Fast X-ray Transient systems \ignore{(\textsc{sfxt}s) }in 2005 \citep{Sguera2005,Sguera2006}, whose evolutionary relation to Sg\textsc{xb}s remains unclear, did little to clarify the situation and much to pave the way to new models which need to be put to the numerical test. Since my Master thesis in 2013, I have made the most of this junction between a harvest of new wind accreting X-ray binaries and the unprecedented computational power entailed by parallel computing to identify the conditions favourable to the formation of a disc around the accretor. Expected to be different from Shakura \& Sunyaev's $\alpha$-disc model \citep{Shakura1973} which fits well the \textsc{rlof} formed discs, the wind-capture discs may be a fertile ground for new kinds of instabilities involving, for instance, torque reversals.\\
\begin{center}
\rule{0.25\textwidth}{1pt}
\end{center}
\newpage
\indent \indent The initial work I realized with Fabien \textsc{Casse} was to provide the hydrodynamical code {\sc mpi-amrvac} (for Message Passing Interface - Adaptive Mesh Refinement Versatile Advection Code, see \cite{Porth:2014wv}) with a robust numerical setup of a planar supersonic flow being gravitationally deflected by a point-mass accretor, the \textsc{Bondi-Hoyle} (\textsc{b-h}) flow \cite{Hoyle:1939fl,Bondi1944}. \ignore{The first version of {\sc mpi-amrvac}, \textsc{vac}, was developed in the 90's by Gabor \textsc{T\'oth} and Rony \textsc{Keppens}. }{\sc mpi-amrvac} is a parallelized Fortran / Open{\sc mpi} set of modules to solve the conservative form of the equations of hydrodynamics and magnetohydrodynamics (\textsc{mhd}) - in a classical or in a relativistic framework - on a grid whose dimensionality can conveniently be modified thanks to the \textsc{lasy} syntax and its Perl preprocessor \citep{Toth1997}. I modified the geometry of the mesh to enforce a constant aspect ration of the cells all over the grid and adapted the \textsc{tvdlf}\footnote{For Total Variation Diminishing - Lax Friedrichs.} numerical scheme we were using accordingly \citep{Mignone2014}. Along with this logarithmically stretched mesh, I designed optimal configurations to fit the parallel computing capacities of our local cluster\footnote{At the Fran\c cois Arago Center (\textsc{fac}e) in Paris 7.} and then, when I was granted 300 kh$\cdot$\textsc{cpu} to spend over 6 months, of the national \href{https://www.cines.fr/en/}{\textsc{cines} cluster}. Thanks to this preliminary hardware work, including the use of profiling tools like Vampir, I managed to run high dynamics range simulations, with cell sizes spanning 5 orders of magnitude\footnote{The equivalent of 17 levels of refinement in \textsc{amr} but without introducing sharp discontinuities in the mesh structure.}. Beyond the technical skills I have developed in high performance computing (\textsc{hpc}), from load balancing to scalability and multi-threading, I earned advanced experience in visualization of large data sets in two and three dimensions. I took advantage of the sophisticated features provided by the VisIt software to highlight novel properties of the \textsc{b-h} flows suggested by analytical studies \citep{Foglizzo1996}. The scientific results of this work are described in details in \cite{ElMellah2015}.\\

\ignore{\indent The scientific results of this work are described in details in \cite{ElMellah2015} and are quickly summarized below. The flow which relaxed on our 2.5D spherical grid featured a detached bow shock ahead the accretor, with a hollow conic tail at high Mach numbers and no axisymmetric instabilities (provided physically accurate inner boundary conditions are set up). The dependence of the mass accretion rate on the Mach number of the flow at infinity was confronted to \cite{Foglizzo1996}'s prediction. Indeed, the latter relied on a physically necessary topological property of the sonic surface that we serendipitously witnessed in our simulation : its anchoring into the accretor. The flow in the vicinity of the accretor, 20 to 2,000 times smaller than the distance to the shock front, was also characterized.\\}
\ignore{\indent Given the consistency of those results, we decided to relax the axissymmetric assumption and head for full 3D simulations. It can also be shown that the large dynamics our setup enables us to reach is a prerequisite to consistenly model a \textsc{b-h} flow, from the large accretion scale down to the neighbourhood of the accretor, when its velocity is of the order of $10^3$km$\cdot$s$^{-1}$, as observed in the wind accreting Sg\textsc{xb}\footnote{See regions A \& B in Figure 2 of \cite{ElMellah2015}.}. Such an assessment drove me into studying the \textsc{Roche} lobe of a compact object accreting matter from the wind of its massive stellar companion. This undergoing work I started with Fabien \textsc{Casse} is described in the following section. On the other hand, with Thierry \textsc{Foglizzo}, I am also looking for axisymmetric instabilities in the steady-state flow I obtained. Motivated by the analytic expectations for an advective-acoustic cycle between the front shock and the sonic surface \citep{Foglizzo2009}, I interpolated the relaxed state we got using a much finer grid and relied on a less diffusive numerical scheme, \textsc{tvdmuscl}. The first insights into this previously unseen instability are mentioned in the last section.\\}
\indent I then went beyond the ideal \textsc{b-h} model to capitalize on all these skills and shed light on wind accretion in X-ray binaries. Since the seminal paper by \textsc{Castor, Abbott \& Klein} in 1975 \citep{Castor1975}, isolated massive stars winds have been thoroughly studied and in-depth models have been given reasonable trust \citep{Lamers1999}. However, in a binary system, the \textsc{Roche} potential modulates the orbital dynamics of the wind which finds itself altered. Fruitful results have been obtained for symbiotic binaries in the last years \cite{deValBorro:2009gk} but in high mass X-ray binaries, the similarity between the orbital separation and the acceleration radius makes the problem more challenging. Only a fraction of the stellar wind is expected to be captured by the compact object within its \textsc{Roche} lobe where I turned the spotlight onto ; so as to get physically-motivated outer boundary conditions for full 3D simulations, I designed an integrator to compute the trajectory of test-particles submitted to radiative accelerations induced by line absorption and scattering by free electrons, in a \textsc{Roche} potential. The code I developed provides, for any set of orbital parameters, estimations of the mass and angular accretion rates within the sphere of gravitational influence of the compact object, along with the aforementioned non planar boundary conditions. The promising orbital configurations are presently piped to \textsc{mpi-amrvac} to compute the hydrodynamics evolution of this supersonic inflow.\\
\\
\newpage
\indent On the other hand, with Thierry \textsc{Foglizzo}, I am carrying out a refined study of the axissymmetric stability of the steady-state flow I obtained in \cite{ElMellah2015}. Indeed, the stability of the \textsc{b-h} flow has long been a matter of debate according to the diverging conclusions numerical groups have drawn since the late 80's \cite{Foglizzo2005}. Motivated by the analytic expectations for an advective-acoustic cycle in the cavity delimited by the front shock and the sonic surface \citep{Foglizzo2009}, I interpolated the relaxed state we got using a much finer grid in the central parts and relied on a less diffusive numerical scheme (\textsc{tvdmuscl}). This new setup fits the needs to resolve wavelengths of perturbations corresponding to growth rates high enough for an amplification to take place in the cavity on a computationally affordable number of time steps. First outcomes indicate suggestive breathing modes excited by the interplay between entropic disturbances adiabatically advected inwards and outflowing acoustic waves ; I am currently performing numerical and physical checks to confirm the origin of this instability and to assess its saturation level as a function of the position of the sonic surface.\\
\begin{center}
\rule{0.25\textwidth}{1pt}
\end{center}
~\\

% J. Dexter
\indent Two-dimensional \textsc{b-h} flows have been long known to be the stage of transverse instabilities large enough to lead to the formation of disc-like structures \citep{Blondin2013}. However, the three-dimensional \textsc{b-h} flows turn out to be more stable, including when embedded in a non trivial potential. Instabilities such as the advective-acoustic one still offer a chance to produce large scale perturbations favourable to a winding of the flow. More generally, the setup I have designed and the \textsc{hpc} methods I have got familiar with open the doors to consistent simulations of flows with physically relevant speeds (i.e. $<10^4$km$\cdot$s$^{-1}$) being accreted on to a compact body.\\
\indent Joining the Max Planc Institute for Extraterrestrial Physics would give me the occasion to provide strong arguments for the disc structure and its very existence in the hardly explored realm of wind-formed discs. Furthermore, I could help to investigate the implications for the innermost regions of the flow being accreted on to a black hole, a burning question regarding, for example, the spin measurements \cite{Penna:2010tp} and the incoming observational facilities (in particular the \textsc{gravity} instrument \cite{Eisenhauer2011} and the Event Horizon Telescope \cite{Doeleman2009}). Coupled to a proper ray-tracing code such as \textit{Geokerr} \cite{Dexter2009}, a \textsc{grhd} version of the setup I designed would already give exciting observables concerning the fate of gas falling either on to a stellar mass black hole in a \textsc{Roche} potential or on to a supermassive black hole sipping its turbulent environment \cite{Ruffert1994b}. 

% M. Pessah 
%\indent Two-dimensional \textsc{b-h} flows have been long known to be the stage of transverse instabilities large enough to lead to the formation of disc-like structures \citep{Blondin2013}. However, the three-dimensional \textsc{b-h} flows turn out to be more stable, including when embedded in a non trivial potential. Instabilities such as the advective-acoustic one still offer a chance to produce large scale perturbations favourable to a winding of the flow. In X-ray binaries, the setup I have designed makes it possible to overcome the multi-scale difficulty of stellar winds with believable speeds (i.e. $<10^4$km$\cdot$s$^{-1}$) being accreted on to a compact object. With the \textsc{hpc} methods I have got familiar with, it opens the doors to consistent simulations of wind accretion accounting for additional Physics. Indeed, once a disc starts to form or once the flow approaches the neutron star magnetosphere, \textsc{mhd} considerations will come into play. Models such as one suggested to explain the off-states of Vela X-1 could then be conclusively disentangled.\\
%\indent Joining the theoretical Astrophysics group of the Niels \textsc{Bohr} International Academy would give me the occasion to apply the methods I developed to numerically similar configurations. I could help to put on the shelves multi-scales and multi-physics simulations of plasmas or two-fluids flows and to pinpoint the triggering conditions for global and local instabilities to develop. The renowned accomplishments of the group in \textsc{mhd} models and simulations of accretion discs convince me that, in Copenhagen, I could take part in strengthening and extending our understanding with robust and versatile numerical setups.

% R. Sunyaev 
%\indent Two-dimensional \textsc{b-h} flows have been long known to be the stage of transverse instabilities large enough to lead to the formation of disc-like structures \citep{Blondin2013}. However, the three-dimensional \textsc{b-h} flows turn out to be more stable, including when embedded in a non trivial potential. Instabilities such as the advective-acoustic one still offer a chance to produce large scale perturbations favourable to a winding of the flow. More generally, the setup I have designed and the \textsc{hpc} methods I have got familiar with open the doors to consistent simulations of flows with physically relevant speeds (i.e. $<10^4$km$\cdot$s$^{-1}$) being accreted on to a compact body.\\
%\indent Joining the Max Planc Institute for Extraterrestrial Physics would give me the occasion to provide strong arguments for the disc structure and its very existence in the hardly explored realm of wind-formed discs. Furthermore, I could help to investigate the implications for the innermost regions of the flow being accreted on to a black hole, a burning question regarding, for example, the spin measurements \cite{Penna:2010tp} and the incoming observational facilities (in particular the \textsc{gravity} instrument \cite{Eisenhauer2011} and the Event Horizon Telescope \cite{Doeleman2009}). Coupled to a proper ray-tracing code such as \textit{Geokerr} \cite{Dexter2009}, a \textsc{grhd} version of the setup I designed would already give exciting observables concerning the fate of gas falling either on to a stellar mass black hole in a \textsc{Roche} potential or on to a supermassive black hole sipping its turbulent environment \cite{Ruffert1994b}. 

\newpage

%\newgeometry{left=2cm,right=2cm,top=2.5cm,bottom=2.5cm}
\setlength{\bibsep}{5pt}
\small
\bibliographystyle{plainnat}
\bibliography{/Users/ielm/Documents/Bibtex/research_statement_no_url}

\end{document}
%%%%%%%%%%%%%%%%%  Fin du fichier Latex  %%%%%%%%%%%%%%%%%%%%%%%%%%%%%%

