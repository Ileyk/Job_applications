\documentclass[11pt]{article}


\usepackage[frenchb]{babel}
\usepackage[utf8]{inputenc}
\usepackage[T1]{fontenc}
\usepackage{graphicx}
\usepackage{float}
\usepackage{hyperref}
\usepackage[usenames,dvipsnames]{pstricks}
\usepackage{epsfig}
\usepackage{pst-grad}
\usepackage{pst-plot}

%\usepackage[a4paper,left=1.6cm,bottom=1.6cm]{geometry}
\usepackage[letterpaper,left=1.2cm,bottom=1.2cm]{geometry}
\marginparwidth = 5pt
\textwidth = 530pt
%\usepackage{showframe}

\usepackage{amsmath,amsfonts,amssymb}
\usepackage{xspace}
\usepackage{fancyhdr} 
\pagestyle{fancy}
\pagenumbering{gobble}

\newcommand{\gr}[1]{\textbf{#1}}
\lhead{RESEARCH STATEMENT}
\rhead{Julien Peloton}
\begin{document}
\thispagestyle{empty}
\thispagestyle{fancy}

The main area of my research interests is observational cosmology and in particular analysis and interpretation of the cosmological observations, with specific emphasis on cosmic microwave background data sets. 
The Cosmic Microwave Background (CMB) is a relic light emitted 380,000 years after the Big Bang  and therefore coming from the farthest regions of our Universe. It consequently carries unique information about the Universe's early evolution.

CMB investigations have been and continue to be a very active and exciting area of research, which has excellent potential to impact profoundly our understanding of cosmology and fundamental physics in its broadest sense, providing a new and unique window on the physics of both the early Universe and the growth of its large-scale structure. 
The expected outcome of this research includes constraints on physics at the energy scales $10^{12}$ higher than what has been achieved at LHC, constraints on the absolute scale of neutrino masses and their mass hierarchy, as well as insights into the nature of dark energy. 
These are some of the most thrilling and profound questions of modern cosmology and physics. 
These are also the questions which have been motivating my research
from the beginning of my Master thesis and which I plan to keep on addressing in the future.

The next few years stand out as a particularly suitable moment to carry out this research program. 
Indeed, following up on a decade of intensive work, this year has finally seen a release of the first results from the ground-based efforts, such as {\sc bicep2}, {\sc polarbear}, {\sc sptpol}, or {\sc actpol}, which proved that the sensitivity necessary to start delivering on this long-standing promise has been at last achieved. 
This together with the anticipated release of the nearly full sky, multi-frequency data from the European Space Agency satellite mission, {\sc{Planck}}, in November 2014 heralds the beginning of a particularly exciting and inevitably fruitful period. 
These observational developments have to be matched by sophisticated analysis of the resulting CMB data sets alone and in conjunction with other cosmological and astrophysical probes. This in turn will require new, statistically robust and numerically efficient analysis algorithms and methodologies to enable unlocking the full scientific potential of the data. The development of such novel techniques and their applications to the diverse cosmological and astrophysical data is at the core of my research program for the future.


Since the beginning of my Master thesis work in 2012, I have been actively participating in this kind of research through my involvement in one of the leading experimental efforts, called {\sc polarbear}, targeting the so-called B-mode polarization signal. 
This has allowed me to develop in-depth understanding of the challenges faced by the field and a skill set necessary to address them. 
Notably, I have gained unique experience in all stages of data analysis of the CMB data sets, from the low-level processing of raw data to sophisticated techniques for statistically robust and numerically efficient estimation of the sky maps and  their statistical properties. 
I have successfully applied all this expertise in the analysis of the 1st year of the {\sc polarbear} data as published in 3 recent papers by the collaboration. 
More specifically, I have lead the systematic errors modeling and analysis for the first observational season. I have contributed and co-led the development of a massively parallel data analysis pipeline including low level processing of raw data, map-making and power spectrum estimation and currently I am co-leading the analysis of the 2nd year data set. 
I have hands-on expertise in devising and optimizing operations of the CMB instruments and an extensive background in advanced statistical and data science techniques. 
I have extensive experience in application and development of novel statistical and data analysis methods in the context of the CMB data analysis.
In this context I have set up a collaboration aiming at evaluation of some of the power spectrum techniques in the context of the B-mode power spectrum estimation and from the perspective
of detecting the primordial gravitational wave signal.
%novel numerical methods and tools such as the pure-pseudo techniques devised to minimize the effects of the leakage on the variance of power spectrum estimates and determine the limits on the tensor-to-scalar ratio that could be realistically set by current and forthcoming measurements of the B-mode angular power spectrum.
I am also co-leading the development of an algorithm for performing a novel interpolation of spin-weighted spherical harmonics defined on the sphere. Our technique capitalizes on
state-of-the-art approaches developed in the context of so-called Fast Multipole Methods. This interpolation technique should result in significant speed-ups in computation
of spin-weighted spherical harmonics on irregular grids of points as, for instance, required in CMB lensing simulations and promises to supersedes any of the methods used to date,
enabling massive Monte Carlo simulations and opening the doors to new investigations, which where hitherto either limited in scope or outrightly impossible due the numerical workload they implied.


In my future research  I propose to capitalize on all these skills in the context of the new generation of CMB and cosmological observatories and huge and intricate data sets they are already producing or expected to produce in the near future. 
Berkeley is a renowned worldwide hub for the observational, experimental and data analysis work in these areas and I am therefore particularly excited about the possibility provided by the Owen Chamberlain Fellowship program to carry out the proposed research there. 
Specifically, I would like to continue my very profitable involvement in the {\sc polarbear} program, by getting involved in two forthcoming reincarnations of the current instrument, which are referred to as {\sc polarbear-ii} and {\sc simons array}. 
These two will be deployed within the next 3 years and produce data sets of daunting volumes and complexity. 
Building on my past involvement in the program, I plan on playing a leadership role in the optimization of these instruments, their operations and the follow-up analysis of their data sets. 
The anticipated {\sc polarbear-ii} and {\sc simons array} data sets alone will be already  extremely powerful in constraining the cosmology. 
They will be however even more impressive, when co-analyzed with other data sets, including data from galaxy surveys, \textit{e.g.}, for studying gravitational lensing effects, but also the Planck public data set with unique insights it will offer into the physics of the galactic emissions, which unavoidably contaminate the measured CMB signals. 
I have a strong hands-on experience on cross-correlating cosmological data sets of different origins deriving from my work on the {\sc polarbear} data set. I intend on further 
developing techniques put forward for this purpose and generalize them to the new context as defined by the new, qualitatively and quantitatively, data sets. 
Yet again, Berkeley Lab is a particularly fertile ground, which  has attracted experts working on different cosmological probes as well different experimental and observational
programs, and where such investigations can be consequently best performed.

While considering all these novel and exciting data analysis challenges, I plan on capitalizing on the interdisciplinary skills I have developed during my PhD
and will look for and will develop novel, more efficient and/or statistically robust approaches.
In particular, I will consider novel techniques and methods originating from data science and apply them to the CMB data processing. 
Such techniques have undeniable  potential to be real game-changers in the analysis of the cosmological data sets.
In this context I look forward to collaborating with the members of the Berkeley Institute for Data Science.

I also plan on investigating applicability of the CMB methodologies and techniques in other research areas. 
Again Berkeley stands out in providing numerous opportunities here, which I plan on actively exploring. 
I am particularly excited about total intensity observations of 21 cm signal. This is a quickly growing area of research, which
offers complementary constraints to those expected from the CMB but call for very similar data analysis tools and techniques as the CMB does. 
This provides an opportunity to leverage years of the development in this field to accelerate the discovery in the new context.
There are already numerous experimental efforts, which are currently on different stages of the development, including long-term projects,
such as SKA, which will be in many way an ultimate challenges, but also an entire host of smaller, "pathfinder" projects, including some
with significant Berkeley involvement, e.g., PAPER.

% Specifically, the rea
%
%
%While following a Master program in Physics from Universit\'e Paris XI in France, I had the opportunity to study a year at Imperial College London in UK receiving advanced teachings and working a year in theoretical physics at the Blackett Laboratory. 
%At my return in France, I specialized in cosmology and I joined the team of Dr. Radek Stompor. 
%My main project is the analysis of the data of the {\sc{POLARBEAR}} experiment, with an end-to-end expertise including : raw data calibration, instrumental systematics evaluation, map-making and power spectrum estimation. 
%This analysis led to the first direct measurement of the B-mode polarization power spectrum. 
%I also dedicate a fraction of my time to prepare the future upgrades of the telescope, {\sc{POLARBEAR-2}} and the {\sc{Simons Array}} by optimizing the future campaigns of observation and developing new tools for the data analysis with a particular attention to the foreground emissions characterization and their removal which poses a significant experimental challenge for operating and forthcoming experiments.
%Finally I devote a part of my time on the latest supercomputing platforms to develop and test novel, scalable numerical algorithms adaptable to cutting-edge computer architectures and specificity of the current and future data sets.
%\\
%
%The different aspects of my work give me a broad perspective on upcoming challenges. 
%I have the opportunity to be in a interdisciplinary field, at the intersection between experimental physics, applied mathematics and computer science. 
%The project I will seek thanks to the Miller Fellowship will perpetuate this spirit, by focusing first on the data analysis of the {\sc{POLARBEAR}} experiment and its upgrades. 
%The knowledge acquiring during my studies will help to intensify the ongoing data analysis and to prepare the future challenges, such as bigger and more complex data sets, with many frequency channels. 
%In the meantime, I will develop and lead different projects including for instance the cross-correlations between different data sets. 
%With the imminent {\sc{Planck}} data release, there will be an opportunity to exploit my knowledge acquired during my studies on foregrounds and component separation to clean the future data sets. 
%On a broader perspective for an application, the comprehension of foreground emissions and their cleaning or separation should be of interest for numerous area of research including the 21 cm total intensity measurements.
%Finally, I will continue my efforts to develop new numerical methods and tools to answer present and future challenges from the community with help of supercomputers.
\end{document}